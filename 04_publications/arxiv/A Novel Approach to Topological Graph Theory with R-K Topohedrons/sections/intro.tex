\section{Introduction}
Since the advent of the "next-generation" high-throughput big-data revolution over the last decade, there has been an explosion of available scientific, business and social-media data, accelerating research, unprecedentedly in most domains of human society. These exponential advances in processing power coupled with distributed cloud-based file storage systems have revealed paradigm shifting implications in science and technology when aided with the correct, adaptable and scalable methods of analytics. Unfortunately figuring out the correct methodology with compatible robustness and scale has been a persistent challenge for researchers and analysts alike, due to the ever-growing size and complexity of high dimensional data sets. In fact, the very perplexing nature of scientific big-data coupled with their astronomical sizes are posing constant challenges to traditional computational methods, which are largely based on combinatorial mathematics and clustering techniques. In some cases, the nature of existing data is not suited to current approaches (for instance, the continuous nature of real-time data differentiation is not suited to conventional clustering methods); in others, the enormous size makes the analysis infeasible with current computing resources. Therefore, it is evident that new computational approaches are required to boost existing Machine Learning (ML) driven analytical tools, systems and products to address these pertinent challenges.

Graph Theory (GT) and Topological Data Analysis (TDA) have recently emerged as independent novel frameworks for extracting hidden meaning and underlying insights from the study of geometric structure, shape and connections of such vast and complex datasets. However modern computational tools lack the technology, efficiency and flexibility to consistently carry out Graph Theory Network Analysis with hierarchical connections with localised clustering due to the inherent variability that could encode directed relationships in the affine connexions of the datasets to build homotopic manifolds and simplicial complexes. They also lack a consistent framework to mathematically define and classify the global properties of the same network through an effective means to smoothly transit between Graph and Topological structures without having to regenerate the entire data geometry from scratch due to lack of persistent homology between the two models. 

Being able to showcase TDA and GT capabilities via smooth mathematical transformations on the same data network without the necessity to recreate its underlying geometric structure encompasses an enormous field of untapped potential in modern scientific big-data analytics. This academic paper explores that very possibility of consistently improving existing GT and TDA technologies with enhanced geometric compatibility while preserving their respective mathematical properties through simple Vectorized Associations in Topological Phase Space. This research also aims to facilitate a smooth transition between these two advanced analytical methodologies through a novel ML driven computational framework by building Self-Expressive Homotopic Topohedrons. These are shown as a special category of 3D Polyhedrons that maintain persistent Homology when projected onto a Topological Phase Space. These special Topohedrons are generated via select, filter-based ML driven optimizations on the underlying n-dimensional data set, preserved within the supressed Topological Space. This work also discusses the conditions involved with the preservation of Homotopy of such Topohedrons under continuous deformations brought about by any changes in Topological Network Entropy and formulates those implications through mathematical and computational models.

The formulation of this work could have seminal implications on high-dimensional, complex scientific data sets especially in the fields of Astronomy and Particle Physics without the necessity of conventional, cumbersome clustering and binning techniques. It also replaces the existing Mapper algorithms with a holistic analytical framework that goes well beyond partial clustering and persistent homology for shape rendering and Isometric Data Compression native to conventional Topological Analytics.




\subsection{Background and Motivations}
The field of Topological Data Analysis is an emergent field with promising techniques for data compression and discovery. Topological Data Analysis is intended to help provide a toolset capable of exposing relevant features from high dimensional data by using geometric concepts. While geometric interpretation is a century old problem, modern conceptions of topological data analysis originated in 2002 with the work of Edelsbrunner et. al and have been considerably contributed to since then. Most notably, Professor Gunner Carlsson pioneered critical work in topological data computation in 2009. Since then it has been a growing field of research vaired set of research focuses.

Most of topological data analysis is centered around proving analysts with a \textit{toolset} to understand fundamental geometric properties of high dimensional data. There are many challenges with current topological data analysis methods and in this paper we tackle the issue of persistence and dynamic systems. By combining concepts of phase space, topology, and graph together we propose a unified framework for analyzing dynamic systems. Our hope is that with future research using this paper as a launching point, researchers will have access to unparrelled flexibility in data analysis, by allowing them to have stable geometric analysis under both relational graph structures and geometric tensors. To implement such a radically new way of approaching data analysis, we have had to introduce a number of concepts such as "Association Vectors", "Roy Simplicies + Complexes", "Roy Topohedrons". Using the flow of information and entropic measurements, we are able to monitor and evaluate the development of our system. We describe the full system as the "Roy-Kesselman Model". 

Details will be elaborated in the following sections however here is the coarse outline of our paper:

\begin{enumerate}
\item Topological Data Analysis - We describe current techniques as well as give a brief literature review.
\item Phase Space - We discuss phase space and it's implication on dynamic systems.
\item Pipeline - The main section of our paper with our pipeline and methodology for implementing our algorithm over dynamic systems.
\item Discussion - We will discuss implications for future work and improvements. 
\end{enumerate}
