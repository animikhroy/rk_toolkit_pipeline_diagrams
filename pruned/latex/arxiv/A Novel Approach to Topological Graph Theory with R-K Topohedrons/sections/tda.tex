\section{Topological Data Analysis: Literature Review}
The field of Topological Data Analysis is an emergent field with promising techniques for data compression and discovery. Generally the pipeline for topological data analysis techniques follow the following pipeline \footcite{Michel2017}:
\begin{enumerate}
\item Input is a finite set of points with a notion of similarity or distance between them. It is generalized metric space of distance, and can be either induced on inherent. 
\item A "countinous" shape is built on the data to highlight topology. They are built by covers over the input matrix and generate a group of simplicial complexes (called a filtration) that reflects the structure of the data at seperate scales.
\item Toplogical information is built from the data.
\item Analysis is applied over the topological set formed by the compression techniques.
\end{enumerate}

There are a variety of challenges in topological data analysis, such as efficient convergence toward reebs graphs, sensitivity to resolutions and filtration, probabilistic interpretations, robust methods that are insentive to input type, flexibility for switching between relational and topological frameworks, persistence problems, and many more. We attempt to immediately address a few of these problems, mainly fusion of graph space and topological space, stability under perterbation, and dynamic systems.
