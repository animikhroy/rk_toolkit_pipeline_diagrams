\section{Topological Data Analysis}
Geometrical and topological approaches to Big Data \footcite{Leonard2016}
TDA \footcite{Wasserman2016}

\subsection{Overview}

\begin{definition}[Topological Space]
A topological space is a set of $X$ along with collection of subsets of $X$ which satisfy the following criteria:
\begin{enumerate}
\item The $\emptyset$ and $X$ are open.
\item The $\cup$ of open sets are open.
\item the $\cap$ of finite open sets are open.\footcite{Childs2016}
\end{enumerate}
\end{definition}
N-dimensional $Hausdorff$ space or $T_2$ is a topological space of $X$ given $a,b \in X$ are singletons and that $U_a, U_b$ are disjoint open sets that contain both a and b.

\subsection{Isometric Compression of High Dimensional Space}
Limit Points
Compactness
Connectedness

\subsection{Graph}
Classical graph theory defines a graph to be an object consistent of a set of vertices $V$ and a set of edges $E$ where an edge is a connected component between 2 vertices.

Thus $G(V,E)$ is consistent of two sets $\{v_1, v_2, ... v_n\}$ and $\{e_1, e_2,...e_n\}$. Cycles, which are a dominate topic in combinatorial topology and graph theory are a sequence of vertices which $v_0 = v_n+1$ consistute a loop sequence. We will not go into detail regarding cycles in this writeup, however considerable work has been done which is relevant for future implications of our work.    

\begin{definition}[Classical Definition of Graph]
A graph $G=(V,E)$ where V and E are a set of vertices and edges, respectfully. An edge is a connected component between two vertices. One of the critical properties that arose out of research by Antoine Vella is that finite graph theoretic paths are consistent with topological path objects. \footcite{Vella2015}  Alternatively, graphs are a 1-complex which is a non-empty set $X$ with a mapping $i:X \rightarrow X$ and an idempotent map $s: X \rightarrow X^0$. \footcite{Everitt2018}
\end{definition}

According to the Diestal and Kuhn theorem, it was proven a graph is a 1 complex which topological ends correspond to graph-theoretical ends $\omega$  which are not dominated by an infinite degree vertex. This results in a clear injection from $\epsilon \rightarrow \omega_{e}$. \footcite{Kuhn2002}. A graph $G$ can be embedded onto a surface $S$ if there is a drawing on $G$ and $S$ which allows the edges to only intersect at endpoints. \footcite{Childs2016}  Moreover, it has been proven that any 1-simplex graph with a non-dominating vertex can be embedded as a planar graph, which in turn is homeomorphic to a sphere. \footcite{Childs2016}

\subsubsection{Graph Properties}

\begin{definition}[Complexity Formula]
  The complexity of a graph is denoted by $|G| + |E|$. \footcite{Verdiere2016}
\end{definition}



The complexity of a graph is denoted by $G = (V,E)$ which $|G| + |E| = $ Complexity  \cite{Verdiere2016} while the Euler Characteristic is defined  by the formula EulerCharacteristic(G) $= |F| - |E| + |V|$ where |F| is the number of faces of a surface. \cite{Childs2016}.

Canonical*

An end of a graph is an important component of a graph that has been discussed in detail by Diestal and Kuhn. 


\subsection{Graph-Toplogical Convergence}
\begin{proposition}
A graph is topologically convergent $iff$ 
\end{proposition}


\subsection{Surfaces}

\begin{definition}[Euler Characteristic]
  We denote the Euler characteristic by $|F| - |E| - |V|$ which is a topologically invariant measurement \footcite{Childs2016} that describes graph embeddings and surface classification. Moreover, an interesting characterisitc of Euler's formula is that all cellular embeddings in $\mathbb{S}^2$ the Euler Characterstic is always equal to 2 \footcite{Childs2016}
 \end{definition}

A surface is a connected 2-manifold such as a sphere or $\mathbb{R}^2$ and has a neighborhood homeomorphic to a torus $\{(x,y) \in \mathbb{R}^2 | x^2 + y^2 + 1\}$. \footcite{Verdiere2016} \footcite{Childs2016}. Faces of a graph embedding can be defined by $S\\G$ where $S\\G$ is a set of discontinous regions. \footcite{Childs2016}. Graph embeddings onto surfaces have a number of unique properties, including curvature.

Surfaces can have One of the benefits of a graph 

\subsection{Persistent Homology}
The previous paragraph outlines the importance of understanding the structure of the phase space. Topology, a mathematical field developed in the last two centuries, provides the necessary tools for that purpose. Topology studies topological features of spaces: namely, properties preserved under continuous deformations of the space, like the number of connected components, loops, or holes. To that end, in algebraic approaches to topology it is a common practice to replace the original space by a simpler one, known as a simplicial complex, containing the same topological features as the original space. 

\begin{definition}
A Simplicial Complex is a generalization of a network that, apart from nodes and edges, contains triangles, tetrahedrons and higher dimensional polytopes. These shapes are known as simplices.
\end{definition}
 The robust mathematical properties of simplicial complexes allow for the implementation of algebraic operations to identify and classify the topological features of the space. These can be arranged in mathematical structures known as homology groups the kth homology group of a space classifies inequivalent (in the sense of being impossible to continuously deform one into another) k-1 dimensional voids of the space.

\subsection{Paths}
A closed path on a surface that perserves orientation is called an $orientation-perserving$ path and $orientable$ if the surface is stable in orientation perservation. \footnote{Combinatorial Topology}

Proof that compact surfaces is homeomorphic to a sphere, gholed torus, or the connective sum of projective planes. 

\subsection{Simplicial Complexes}
Talking Local Behavior. Simplicial Complexes show glocal shape in local constraints. \footcite{Zomorodian2008}
\subsection{Mapper}
An overview of the mapper algorithm \footcite{Mapper}
\subsubsection{Topological Motivations}
Mapper \footcite{Mapper}
Convergence to Reebs Space \footcite{Wang2016}
Surface Reconstruction \footcite{Watson2010}
Surface Reconstruction \footcite{Kimia2008}

Issues with Cell to Graph
\begin{itemize} 
\item{A one dimensional cell complex isn't a graph}
\item{Topological Structure is much more sophisiticated than Graph Structure. Cardinalities differ and Cell Complex is often described as injection of unit interval into Hausdoff space. (Page 2)\footcite{Vella2015}}
\item {Ground Sets are entirely different between graph and topology  (Page 1)}
\item {Topological Consistency (Page 2)}
\item {Simplicial Complex is a homemorph of a circle of a cycle graph. Impossible to recover because verices are topologically identical so forced to retain combinatorial information.}
\end{itemize}
Findings of Vella
\begin{itemize}
\item{Classical graph paths are a specific case of topological objects}
\item{Cycle spaces and bondspaces for compact weak Hausdorff spaces are orthogonal and generated by bonds and cycles}
\item{fern dendrites are arcwise connected}
\end{itemize}

\footnote{Lideoff Space https://dantopology.wordpress.com/2012/04/29/elementary-examples-of-lindelof-spaces-and-separable-spaces/}


\subsubsection{Overview of Algorithm}
Algorithm Overview  \footcite{Muller}
More on the Mapper Algorithm \footcite{Stovner}
\subsubsection{Loss of Relations in Mapper}
Mapper on Graphs for Relationship Preserving Clustering \footcite{MOG}
\subsubsection{Arbitrary Binning}
Envasion paths in mobile sensor networks \footcite{Carlsson2015}
\subsubsection{Choice Paradox}
An Introduction to Computational Topology \footcite{Adams}
\footnote{https://www.kdnuggets.com/2018/01/topological-data-analysis.html}
Clustering: Reeb graphs, Morse-Smale clustering \footnote{Gerber, S., Rübel, O., Bremer, P. T., Pascucci, V., and  Whitaker, R. T. (2013). Morse–smale regression. Journal of Computational and Graphical Statistics, 22(1), 193-214.}, or Mapper clustering
\footnote{Farrelly, C. M., Schwartz, S. J., Amodeo, A. L., Feaster, D. J., Steinley, D. L., Meca, A., and Picariello, S. (2017). The analysis of bridging constructs with hierarchical clustering methods: An application to identity. Journal of Research in Personality, 70, 93-106.}

\subsubsection{Sensitivity Issues}
Shape of an Image \footcite{Rosen2017}
Coorindate-fre coverage in sensor networks with controlled boundary \footcite{Ghrist2006}

